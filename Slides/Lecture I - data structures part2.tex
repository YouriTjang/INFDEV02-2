\documentclass{beamer}
\usetheme[hideothersubsections]{HRTheme}
\usepackage{beamerthemeHRTheme}
\usepackage[utf8]{inputenc}
\usepackage{graphicx}
\usepackage{MnSymbol}

\title{Data structures}

\author{TEAM INFDEV}

\institute{Hogeschool Rotterdam \\ 
Rotterdam, Netherlands}

\date{}

\begin{document}
\maketitle

\SlideSection{Designing data structures}
\SlideSubSection{Semantics of Python classes}
\begin{slide}{
\item The semantics of Python classes require a more sophisticated model of memory
\item Memory is now divided in two
\begin{description}
\item[\textbf{STACK}] The state that we used so far, for primitive values (\texttt{int}, \texttt{string}, etc.)
\item[\textbf{HEAP}] A storage for complex values such as classes
\end{description}
}\end{slide}

\begin{slide}{
\item An instruction \texttt{I} will now transform the initial heap and stack \texttt{H,S} into the resulting (possibly changed) heap and stack \texttt{H',S'} \footnote{in addition to the program counter \texttt{PC}, which always behaves in the same way}

$$<PC,H,S> \overset{I}{\rightarrow} <PC',H',S'>$$
}\end{slide}

\SlideSubSection{Semantics of creation}
\begin{slide}{
\item Consider creation of a Python class: \texttt{x = <<Name>>(...)} (shortened to \texttt{xName})
\item This affects both memories
\begin{description}
\item[\textbf{HEAP}] We create and initialize a new instance of class \texttt{<<Name>>}
\item[\textbf{STACK}] We add an entry \texttt{x} to the stack, which references to the newly created instance
\end{description}
}\end{slide}

\begin{slide}{
\item Given that:
\item $|H|$ is the size of the heap at creation, which we call the \textbf{address} of the new instance
\item $\llangle Name \rrangle(...)$ is a new instance of the class, which contains a map from the attribute names to their values

\fontsize{8pt}{7.2}\selectfont
$$<PC,H,S> \overset{xName}{\rightarrow} <PC+1,H[|H| \mapsto \llangle Name \rrangle(\dots)],S[x \mapsto |H|]>$$

\item \texttt{x} is, unsurprisingly, called a \textbf{reference}
\begin{itemize}
\item it does not contain the value of the class instance
\item it merely tells us where to find it
\end{itemize}
}\end{slide}

\SlideSubSection{Attribute lookup}
\begin{slide}{
\item Consider reading an attribute (also called lookup)
\item \texttt{x.<<A>>} (shortened to \texttt{xA})
\item Where is it in memory?
\begin{description}
\item[\textbf{STACK}] We find an entry \texttt{x}, which tells us where the corresponding instance of the class is found
\item[\textbf{HEAP}] We find the actual attribute in the map of attributes
\end{description}

$$<PC,H,S> \overset{xA}{\rightarrow} <PC+1,H[S[x]][\llangle A \rrangle],S>$$
}\end{slide}



\SlideSubSection{Attribute update}
\begin{slide}{
\item Consider assigning to an attribute
\item \texttt{x.<<A>> = v} (shortened to \texttt{xAv})
\item Where is it in memory?
\begin{description}
\item[\textbf{STACK}] We find an entry \texttt{x}, which tells us where the corresponding instance of the class is found
\item[\textbf{HEAP}] We reassign the actual attribute in the map of attributes
\end{description}

$$<PC,H,S> \overset{xAv}{\rightarrow} <PC+1,H[S[x] \mapsto H[S[x]][A \mapsto v]]$$
% TODO add "how to read it"
}\end{slide}

\SlideSubSection{Examples}
\begin{slide}{
\item We can now implement our player data type
\item We will use a Python class to do so
\item We will then create concrete instances of it, and use them
}\end{slide}

\begin{frame}[fragile]{The blueprint to implement}
\begin{lstlisting}
Abstraction Player =
  Name, which is a string
  PositionX, which is a number
  PositionY, which is a number
\end{lstlisting}
\end{frame}

\begin{frame}[fragile]{The implemented class}
\begin{lstlisting}
class Player:
  def __init__(self, name, posX, posY):
    self.Name = name
    self.PositionX = posX
    self.PositionY = posY
\end{lstlisting}
\end{frame}

\begin{frame}[fragile]{Creating concrete instances}
\begin{lstlisting}
playerOneName = "P1"
playerOnePositionX = 0.0
playerOnePositionY = 0.0

playerTwoName = "P2"
playerTwoPositionX = 5.0
playerTwoPositionY = 0.0

playerThreeName = "P3"
playerThreePositionX = 10.0
playerThreePositionY = 0.0
\end{lstlisting}

Becomes:

\begin{lstlisting}
playerOne   = Player("P1", 0.0, 0.0)
playerTwo   = Player("P2", 5.0, 0.0)
playerThree = Player("P3", 10.0, 0.0)
\end{lstlisting}
\end{frame}



\begin{frame}[fragile]{Creating concrete instances}
\begin{memorytable}
{|c|}
{PC}
{1}
{|c|}
{}
{}
\end{memorytable}

\ \\

\begin{lstlisting}
playerOne   = Player("P1", 0.0, 0.0)
playerTwo   = Player("P2", 5.0, 0.0)
playerThree = Player("P3", 10.0, 0.0)
\end{lstlisting}

\pause

\begin{memorytable}
{|c|c|}
{PC & playerOne}
{2 & ref(0)}
{|c|}
{0}
{[N $\mapsto$ "P1"; PX $\mapsto$ 0.0; PY $\mapsto$ 0.0]}
\end{memorytable}
\end{frame}

\begin{frame}[fragile]{Creating concrete instances}
\begin{memorytable}
{|c|c|}
{PC & playerOne}
{2 & ref(0)}
{|c|}
{0}
{[N $\mapsto$ "P1"; PX $\mapsto$ 0.0; PY $\mapsto$ 0.0]}
\end{memorytable}

\ \\

\begin{lstlisting}
playerOne   = Player("P1", 0.0, 0.0)
playerTwo   = Player("P2", 5.0, 0.0)
playerThree = Player("P3", 10.0, 0.0)
\end{lstlisting}

\pause

\begin{memorytable}
{|c|c|c|}
{PC & playerOne & playerTwo}
{3 & ref(0) & ref(1)}
{|c|c|}
{0 & 1}
{... & [N $\mapsto$ "P2"; PX $\mapsto$ 5.0; PY $\mapsto$ 0.0]}
\end{memorytable}
\end{frame}

\begin{frame}[fragile]{Creating concrete instances}
\begin{memorytable}
{|c|c|c|}
{PC & playerOne & playerTwo}
{3 & ref(0) & ref(1)}
{|c|c|}
{0 & 1}
{... & [N $\mapsto$ "P2"; PX $\mapsto$ 5.0; PY $\mapsto$ 0.0]}
\end{memorytable}

\ \\

\begin{lstlisting}
playerOne   = Player("P1", 0.0, 0.0)
playerTwo   = Player("P2", 5.0, 0.0)
playerThree = Player("P3", 10.0, 0.0)
\end{lstlisting}

\pause

\begin{memorytable}
{|c|c|c|c|}
{PC & playerOne & playerTwo & playerThree}
{4 & ref(0) & ref(1) & ref(2)}
{|c|c|c|}
{0 & 1 & 2}
{... & ... & [N $\mapsto$ "P3"; PX $\mapsto$ 10.0; PY $\mapsto$ 0.0]}
\end{memorytable}
\end{frame}

\begin{frame}[fragile]{Using the concrete instances}
Suppose we wish to access \texttt{playerOne.PositionX}

\begin{memorytable}
{|c|c|c|c|}
{PC & playerOne & playerTwo & playerThree}
{4 & ref(0) & ref(1) & ref(2)}
{|c|c|c|}
{0 & 1 & 2}
{[N $\mapsto$ "P1"; PX $\mapsto$ 0.0; PY $\mapsto$ 0.0] & ... & ...}
\end{memorytable}

\pause

First we look in the stack:

\begin{memorytable}
{|c|c|c|c|}
{PC & playerOne & playerTwo & playerThree}
{5 & \red{ref(0)} & ref(1) & ref(2)}
{|c|c|c|}
{0 & 1 & 2}
{[N $\mapsto$ "P1"; PX $\mapsto$ 0.0; PY $\mapsto$ 0.0] & ... & ...}
\end{memorytable}
\end{frame}

\begin{frame}[fragile]{Using the concrete instances}
Suppose we wish to access \texttt{playerOne.PositionX}

\begin{memorytable}
{|c|c|c|c|}
{PC & playerOne & playerTwo & playerThree}
{5 & \red{ref(0)} & ref(1) & ref(2)}
{|c|c|c|}
{0 & 1 & 2}
{[N $\mapsto$ "P1"; PX $\mapsto$ 0.0; PY $\mapsto$ 0.0] & ... & ...}
\end{memorytable}

\pause

Then we look in the heap:

\begin{memorytable}
{|c|c|c|c|}
{PC & playerOne & playerTwo & playerThree}
{5 & ref(0) & ref(1) & ref(2)}
{|c|c|c|}
{0 & 1 & 2}
{\red{[N $\mapsto$ "P1"; PX $\mapsto$ 0.0; PY $\mapsto$ 0.0]} & ... & ...}
\end{memorytable}
\end{frame}

\begin{frame}[fragile]{Using the concrete instances}
Suppose we wish to access \texttt{playerOne.PositionX}

\begin{memorytable}
{|c|c|c|c|}
{PC & playerOne & playerTwo & playerThree}
{5 & ref(0) & ref(1) & ref(2)}
{|c|c|c|}
{0 & 1 & 2}
{\red{[N $\mapsto$ "P1"; PX $\mapsto$ 0.0; PY $\mapsto$ 0.0]} & ... & ...}
\end{memorytable}

\pause

Finally we search the right attribute (\texttt{PositionX}):

\begin{memorytable}
{|c|c|c|c|}
{PC & playerOne & playerTwo & playerThree}
{5 & ref(0) & ref(1) & ref(2)}
{|c|c|c|}
{0 & 1 & 2}
{[N $\mapsto$ "P1"; \red{PX $\mapsto$ 0.0}; PY $\mapsto$ 0.0] & ... & ...}
\end{memorytable}
\end{frame}

%%% Aliasing of references

\SlideSection{Designing data structures}
\SlideSubSection{Are we there yet?}
\begin{slide}{
\item We can keep extending our knowledge about the problem
\item For example, we might notice that \texttt{PositionX} and \texttt{PositionY} might happen in other places of the program
\item \textbf{What could we do?}
\pause
\item We could define a \texttt{Point2D} (or \texttt{Vector2D}) data structure!
}\end{slide}

\SlideSubSection{Refined data structures}
\begin{frame}[fragile]{Refined data structures}
\begin{lstlisting}
class Vector2:
  def __init__(self, x, y):
    self.X = x
    self.Y = y
    
class PlayerRefined:
  def __init__(self, name, posX, posY):
    self.Name = name
    self.Position = Vector2(posX,posY)
\end{lstlisting}
\end{frame}

\begin{slide}{
\item Creation is precisely identical to the previous sample
\item The \texttt{\_\_init\_\_} of the \texttt{PlayerRefined} has the same inputs
\item Where we had \texttt{playerOne = Player("P1", 0.0, 0.0)}
\item Now we have \texttt{playerOne = PlayerRefined("P1", 0.0, 0.0)}
}\end{slide}

\begin{slide}{
\item Usage of the new player definition is almost identical to the previous
\item Only changes are lookups like: \texttt{playerOne.PositionY}
\item \textbf{What do they become now?}
\item \texttt{playerOne.Position.Y}
}\end{slide}

\begin{slide}{
\item What does memory look like now with a player that contains a vector?
\item Stack is similar to previous instance
\item Heap contains a reference to a vector!
}\end{slide}

\begin{frame}[fragile]{Creating concrete instances}
\begin{memorytable}
{|c|}
{PC}
{1}
{|c|}
{}
{}
\end{memorytable}

\ \\

\begin{lstlisting}
playerOne   = PlayerRefined("P1", 0.0, 0.0)
\end{lstlisting}

\pause

\begin{memorytable}
{|c|c|}
{PC & playerOne}
{2 & ref(0)}
{|c|c|}
{0 & 1}
{[N $\mapsto$ "P1"; P $\mapsto$ ref(1);] & [X $\mapsto$ 0.0; Y $\mapsto$ 0.0]}
\end{memorytable}
\end{frame}

\SlideSubSection{What characterizes a good design of data structures?}
\begin{slide}{
\item \textbf{Reuse} of code in places where otherwise repetition would happen
\item \textbf{Encapsulation} of the semantics of the data structure
\item \textbf{Loose coupling} between the data structure and the rest of the program
}\end{slide}

\SlideSubSection{Reuse of code}
\begin{slide}{
\item Repetition is dangerous
\item A small change in one place but not in the others can lead to unexpected consequences
\item More code to read means more mental overhead
\item Actual work of the program is hidden under lots of noise and thus less visible
}\end{slide}

\SlideSubSection{Encapsulation}
\begin{slide}{
\item A data structure has a single, clear, well-defined goal
\item Its name clearly explains what it contains and does
\item There is no multiple functionality mix
\pause
\item It's a cold beer, not a cocktail
}\end{slide}

\SlideSubSection{Loose coupling}
\begin{slide}{
\item A data structure is a closed and complete unit
\item To use it, you just need to declare it and initialize it
\item The rest of the program integrates a well-designed data structure with minimal modification
}\end{slide}

\SlideSubSection{How do we verify all this?!?}
\begin{slide}{
\item Takes experience and good taste
\item It is an old story
\item Remember: you have the power to make your own life a living Hell...
\pause
\item ...unless you reason first and write code after
}\end{slide}

\SlideSection{Assignment}
\SlideSubSection{Build, in class, a series of data structures}
\begin{slide}{
\item Tyre
\item Wheel
\item Engine
\item Seat
\item Light
\item Person (driver and passenger)
\item Car
}\end{slide}

\SlideSection{Conclusion}
\SlideSubSection{Lecture topics}
\begin{slide}{
\item Abstraction is the fundamental mechanism that allows us to group concepts together and refer to them as if they were a single concept
\item For example, a name and two numbers became a \texttt{player}
\item We then use the new concept (the \texttt{player}) without having to explicitly mention all of its components every time
\item This makes it leaner for us to manipulate complex programs, as less concepts (``actors'') make an appearance
}\end{slide}

\begin{thankyou}
\end{thankyou}

\end{document}

\begin{slide}{
\item ...
}\end{slide}

\begin{frame}[fragile]
\begin{lstlisting}
...
\end{lstlisting}
\end{frame}

\begin{frame}[fragile]
\begin{codewithblock}{\item $x^y + \sum_i i^2$}
if x > y then 
  print x 
else 
  print y
\end{codewithblock}
\end{frame}
